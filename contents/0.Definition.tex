\section{Important definitions}
\begin{definition}[Phép biến đổi tự nhiên]
    Cho $F_1, F_2: \mathcal{C} \to \mathcal{D}$ là hai hàm tử \textbf{hiệp biến}. Một \textit{phép biến đổi tự nhiên} $\phi:F_1\to F_2$ gồm một họ các cấu xạ $\phi_X \in \Hom(F_1(X),F_2(X)),~X \in \mathcal{C} $ sao cho các biểu đồ tương ứng sau giao hoán với mọi $X, Y \in \Ob(\mathcal{C})$
    \[\begin{tikzcd}[ampersand replacement=\&,cramped]
	{F_1(X)} \&\& {F_1(Y)} \\
	\\
	{F_2(X)} \&\& {F_2(Y)}
	\arrow["{F_1(\alpha)}", from=1-1, to=1-3]
	\arrow["{\phi_X}"', from=1-1, to=3-1]
	\arrow["{\phi_Y}", from=1-3, to=3-3]
	\arrow["{F_2(\alpha)}"', from=3-1, to=3-3]
\end{tikzcd}\]
Tương tự nếu $F_1,F_2$ là hai hàm tử phản biến
    \[\begin{tikzcd}[ampersand replacement=\&,cramped]
	{F_1(X)} \&\& {F_1(Y)} \\
	\\
	{F_2(X)} \&\& {F_2(Y)}
	\arrow["{\phi_X}"', from=1-1, to=3-1]
	\arrow["{F_1(\alpha)}"', from=1-3, to=1-1]
	\arrow["{\phi_Y}", from=1-3, to=3-3]
	\arrow["{F_2(\alpha)}", from=3-3, to=3-1]
\end{tikzcd}\]
\end{definition}

\begin{example}
    Với $A, B \in \Ob(\mathcal{C})$, xét hàm tử $\Hom(A,-), \Hom(B,-):\mathcal{C} \to \textbf{Sets}$. Khi đó ta có phép biến đổi tự nhiên 
\[\phi: \Hom(A,-) \to \Hom(B,-)\]
    \[\begin{tikzcd}[ampersand replacement=\&,cramped]
	X \&\& {\hom(A,X)} \&\& {\hom(B,X)} \\
	\\
	Y \&\& {\hom(A,Y)} \&\& {\hom(B,Y)}
	\arrow["{\hom(A,-)}", maps to, from=1-1, to=1-3]
	\arrow["\alpha"{description}, dashed, from=1-1, to=3-1]
	\arrow["{\phi_X}", from=1-3, to=1-5]
	\arrow["{\hom(A,\alpha)}"{description}, dashed, from=1-3, to=3-3]
	\arrow["{\hom(B,\alpha)}"{description}, dashed, from=1-5, to=3-5]
	\arrow["{\hom(A,-)}", maps to, from=3-1, to=3-3]
	\arrow["{\phi_Y}", from=3-3, to=3-5]
\end{tikzcd}\]

\end{example}